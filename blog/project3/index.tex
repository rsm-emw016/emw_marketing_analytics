% Options for packages loaded elsewhere
\PassOptionsToPackage{unicode}{hyperref}
\PassOptionsToPackage{hyphens}{url}
\PassOptionsToPackage{dvipsnames,svgnames,x11names}{xcolor}
%
\documentclass[
  letterpaper,
  DIV=11,
  numbers=noendperiod]{scrartcl}

\usepackage{amsmath,amssymb}
\usepackage{iftex}
\ifPDFTeX
  \usepackage[T1]{fontenc}
  \usepackage[utf8]{inputenc}
  \usepackage{textcomp} % provide euro and other symbols
\else % if luatex or xetex
  \usepackage{unicode-math}
  \defaultfontfeatures{Scale=MatchLowercase}
  \defaultfontfeatures[\rmfamily]{Ligatures=TeX,Scale=1}
\fi
\usepackage{lmodern}
\ifPDFTeX\else  
    % xetex/luatex font selection
\fi
% Use upquote if available, for straight quotes in verbatim environments
\IfFileExists{upquote.sty}{\usepackage{upquote}}{}
\IfFileExists{microtype.sty}{% use microtype if available
  \usepackage[]{microtype}
  \UseMicrotypeSet[protrusion]{basicmath} % disable protrusion for tt fonts
}{}
\makeatletter
\@ifundefined{KOMAClassName}{% if non-KOMA class
  \IfFileExists{parskip.sty}{%
    \usepackage{parskip}
  }{% else
    \setlength{\parindent}{0pt}
    \setlength{\parskip}{6pt plus 2pt minus 1pt}}
}{% if KOMA class
  \KOMAoptions{parskip=half}}
\makeatother
\usepackage{xcolor}
\setlength{\emergencystretch}{3em} % prevent overfull lines
\setcounter{secnumdepth}{-\maxdimen} % remove section numbering
% Make \paragraph and \subparagraph free-standing
\makeatletter
\ifx\paragraph\undefined\else
  \let\oldparagraph\paragraph
  \renewcommand{\paragraph}{
    \@ifstar
      \xxxParagraphStar
      \xxxParagraphNoStar
  }
  \newcommand{\xxxParagraphStar}[1]{\oldparagraph*{#1}\mbox{}}
  \newcommand{\xxxParagraphNoStar}[1]{\oldparagraph{#1}\mbox{}}
\fi
\ifx\subparagraph\undefined\else
  \let\oldsubparagraph\subparagraph
  \renewcommand{\subparagraph}{
    \@ifstar
      \xxxSubParagraphStar
      \xxxSubParagraphNoStar
  }
  \newcommand{\xxxSubParagraphStar}[1]{\oldsubparagraph*{#1}\mbox{}}
  \newcommand{\xxxSubParagraphNoStar}[1]{\oldsubparagraph{#1}\mbox{}}
\fi
\makeatother


\providecommand{\tightlist}{%
  \setlength{\itemsep}{0pt}\setlength{\parskip}{0pt}}\usepackage{longtable,booktabs,array}
\usepackage{calc} % for calculating minipage widths
% Correct order of tables after \paragraph or \subparagraph
\usepackage{etoolbox}
\makeatletter
\patchcmd\longtable{\par}{\if@noskipsec\mbox{}\fi\par}{}{}
\makeatother
% Allow footnotes in longtable head/foot
\IfFileExists{footnotehyper.sty}{\usepackage{footnotehyper}}{\usepackage{footnote}}
\makesavenoteenv{longtable}
\usepackage{graphicx}
\makeatletter
\newsavebox\pandoc@box
\newcommand*\pandocbounded[1]{% scales image to fit in text height/width
  \sbox\pandoc@box{#1}%
  \Gscale@div\@tempa{\textheight}{\dimexpr\ht\pandoc@box+\dp\pandoc@box\relax}%
  \Gscale@div\@tempb{\linewidth}{\wd\pandoc@box}%
  \ifdim\@tempb\p@<\@tempa\p@\let\@tempa\@tempb\fi% select the smaller of both
  \ifdim\@tempa\p@<\p@\scalebox{\@tempa}{\usebox\pandoc@box}%
  \else\usebox{\pandoc@box}%
  \fi%
}
% Set default figure placement to htbp
\def\fps@figure{htbp}
\makeatother

\KOMAoption{captions}{tableheading}
\makeatletter
\@ifpackageloaded{tcolorbox}{}{\usepackage[skins,breakable]{tcolorbox}}
\@ifpackageloaded{fontawesome5}{}{\usepackage{fontawesome5}}
\definecolor{quarto-callout-color}{HTML}{909090}
\definecolor{quarto-callout-note-color}{HTML}{0758E5}
\definecolor{quarto-callout-important-color}{HTML}{CC1914}
\definecolor{quarto-callout-warning-color}{HTML}{EB9113}
\definecolor{quarto-callout-tip-color}{HTML}{00A047}
\definecolor{quarto-callout-caution-color}{HTML}{FC5300}
\definecolor{quarto-callout-color-frame}{HTML}{acacac}
\definecolor{quarto-callout-note-color-frame}{HTML}{4582ec}
\definecolor{quarto-callout-important-color-frame}{HTML}{d9534f}
\definecolor{quarto-callout-warning-color-frame}{HTML}{f0ad4e}
\definecolor{quarto-callout-tip-color-frame}{HTML}{02b875}
\definecolor{quarto-callout-caution-color-frame}{HTML}{fd7e14}
\makeatother
\makeatletter
\@ifpackageloaded{caption}{}{\usepackage{caption}}
\AtBeginDocument{%
\ifdefined\contentsname
  \renewcommand*\contentsname{Table of contents}
\else
  \newcommand\contentsname{Table of contents}
\fi
\ifdefined\listfigurename
  \renewcommand*\listfigurename{List of Figures}
\else
  \newcommand\listfigurename{List of Figures}
\fi
\ifdefined\listtablename
  \renewcommand*\listtablename{List of Tables}
\else
  \newcommand\listtablename{List of Tables}
\fi
\ifdefined\figurename
  \renewcommand*\figurename{Figure}
\else
  \newcommand\figurename{Figure}
\fi
\ifdefined\tablename
  \renewcommand*\tablename{Table}
\else
  \newcommand\tablename{Table}
\fi
}
\@ifpackageloaded{float}{}{\usepackage{float}}
\floatstyle{ruled}
\@ifundefined{c@chapter}{\newfloat{codelisting}{h}{lop}}{\newfloat{codelisting}{h}{lop}[chapter]}
\floatname{codelisting}{Listing}
\newcommand*\listoflistings{\listof{codelisting}{List of Listings}}
\makeatother
\makeatletter
\makeatother
\makeatletter
\@ifpackageloaded{caption}{}{\usepackage{caption}}
\@ifpackageloaded{subcaption}{}{\usepackage{subcaption}}
\makeatother

\usepackage{bookmark}

\IfFileExists{xurl.sty}{\usepackage{xurl}}{} % add URL line breaks if available
\urlstyle{same} % disable monospaced font for URLs
\hypersetup{
  pdftitle={Multinomial Logit Model},
  pdfauthor={Your Name},
  colorlinks=true,
  linkcolor={blue},
  filecolor={Maroon},
  citecolor={Blue},
  urlcolor={Blue},
  pdfcreator={LaTeX via pandoc}}


\title{Multinomial Logit Model}
\author{Your Name}
\date{2025-05-19}

\begin{document}
\maketitle


This assignment expores two methods for estimating the MNL model: (1)
via Maximum Likelihood, and (2) via a Bayesian approach using a
Metropolis-Hastings MCMC algorithm.

\subsection{1. Likelihood for the Multi-nomial Logit (MNL)
Model}\label{likelihood-for-the-multi-nomial-logit-mnl-model}

Suppose we have \(i=1,\ldots,n\) consumers who each select exactly one
product \(j\) from a set of \(J\) products. The outcome variable is the
identity of the product chosen \(y_i \in \{1, \ldots, J\}\) or
equivalently a vector of \(J-1\) zeros and \(1\) one, where the \(1\)
indicates the selected product. For example, if the third product was
chosen out of 3 products, then either \(y=3\) or \(y=(0,0,1)\) depending
on how we want to represent it. Suppose also that we have a vector of
data on each product \(x_j\) (eg, brand, price, etc.).

We model the consumer's decision as the selection of the product that
provides the most utility, and we'll specify the utility function as a
linear function of the product characteristics:

\[ U_{ij} = x_j'\beta + \epsilon_{ij} \]

where \(\epsilon_{ij}\) is an i.i.d. extreme value error term.

The choice of the i.i.d. extreme value error term leads to a closed-form
expression for the probability that consumer \(i\) chooses product
\(j\):

\[ \mathbb{P}_i(j) = \frac{e^{x_j'\beta}}{\sum_{k=1}^Je^{x_k'\beta}} \]

For example, if there are 3 products, the probability that consumer
\(i\) chooses product 3 is:

\[ \mathbb{P}_i(3) = \frac{e^{x_3'\beta}}{e^{x_1'\beta} + e^{x_2'\beta} + e^{x_3'\beta}} \]

A clever way to write the individual likelihood function for consumer
\(i\) is the product of the \(J\) probabilities, each raised to the
power of an indicator variable (\(\delta_{ij}\)) that indicates the
chosen product:

\[ L_i(\beta) = \prod_{j=1}^J \mathbb{P}_i(j)^{\delta_{ij}} = \mathbb{P}_i(1)^{\delta_{i1}} \times \ldots \times \mathbb{P}_i(J)^{\delta_{iJ}}\]

Notice that if the consumer selected product \(j=3\), then
\(\delta_{i3}=1\) while \(\delta_{i1}=\delta_{i2}=0\) and the likelihood
is:

\[ L_i(\beta) = \mathbb{P}_i(1)^0 \times \mathbb{P}_i(2)^0 \times \mathbb{P}_i(3)^1 = \mathbb{P}_i(3) = \frac{e^{x_3'\beta}}{\sum_{k=1}^3e^{x_k'\beta}} \]

The joint likelihood (across all consumers) is the product of the \(n\)
individual likelihoods:

\[ L_n(\beta) = \prod_{i=1}^n L_i(\beta) = \prod_{i=1}^n \prod_{j=1}^J \mathbb{P}_i(j)^{\delta_{ij}} \]

And the joint log-likelihood function is:

\[ \ell_n(\beta) = \sum_{i=1}^n \sum_{j=1}^J \delta_{ij} \log(\mathbb{P}_i(j)) \]

\subsection{2. Simulate Conjoint Data}\label{simulate-conjoint-data}

We will simulate data from a conjoint experiment about video content
streaming services. We elect to simulate 100 respondents, each
completing 10 choice tasks, where they choose from three alternatives
per task. For simplicity, there is not a ``no choice'' option; each
simulated respondent must select one of the 3 alternatives.

Each alternative is a hypothetical streaming offer consistent of three
attributes: (1) brand is either Netflix, Amazon Prime, or Hulu; (2) ads
can either be part of the experience, or it can be ad-free, and (3)
price per month ranges from \$4 to \$32 in increments of \$4.

The part-worths (ie, preference weights or beta parameters) for the
attribute levels will be 1.0 for Netflix, 0.5 for Amazon Prime (with 0
for Hulu as the reference brand); -0.8 for included adverstisements (0
for ad-free); and -0.1*price so that utility to consumer \(i\) for
hypothethical streaming service \(j\) is

\[
u_{ij} = (1 \times Netflix_j) + (0.5 \times Prime_j) + (-0.8*Ads_j) - 0.1\times Price_j + \varepsilon_{ij}
\]

where the variables are binary indicators and \(\varepsilon\) is Type 1
Extreme Value (ie, Gumble) distributed.

The following code provides the simulation of the conjoint data.

\begin{tcolorbox}[enhanced jigsaw, toptitle=1mm, bottomtitle=1mm, left=2mm, rightrule=.15mm, leftrule=.75mm, colbacktitle=quarto-callout-note-color!10!white, opacityback=0, opacitybacktitle=0.6, bottomrule=.15mm, titlerule=0mm, colframe=quarto-callout-note-color-frame, breakable, toprule=.15mm, arc=.35mm, title=\textcolor{quarto-callout-note-color}{\faInfo}\hspace{0.5em}{Note}, coltitle=black, colback=white]

\begin{verbatim}
#| eval: false

# set seed for reproducibility
set.seed(123)

# define attributes
brand <- c("N", "P", "H") # Netflix, Prime, Hulu
ad <- c("Yes", "No")
price <- seq(8, 32, by=4)

# generate all possible profiles
profiles <- expand.grid(
    brand = brand,
    ad = ad,
    price = price
)
m <- nrow(profiles)

# assign part-worth utilities (true parameters)
b_util <- c(N = 1.0, P = 0.5, H = 0)
a_util <- c(Yes = -0.8, No = 0.0)
p_util <- function(p) -0.1 * p

# number of respondents, choice tasks, and alternatives per task
n_peeps <- 100
n_tasks <- 10
n_alts <- 3

# function to simulate one respondent’s data
sim_one <- function(id) {
  
    datlist <- list()
    
    # loop over choice tasks
    for (t in 1:n_tasks) {
        
        # randomly sample 3 alts (better practice would be to use a design)
        dat <- cbind(resp=id, task=t, profiles[sample(m, size=n_alts), ])
        
        # compute deterministic portion of utility
        dat$v <- b_util[dat$brand] + a_util[dat$ad] + p_util(dat$price) |> round(10)
        
        # add Gumbel noise (Type I extreme value)
        dat$e <- -log(-log(runif(n_alts)))
        dat$u <- dat$v + dat$e
        
        # identify chosen alternative
        dat$choice <- as.integer(dat$u == max(dat$u))
        
        # store task
        datlist[[t]] <- dat
    }
    
    # combine all tasks for one respondent
    do.call(rbind, datlist)
}

# simulate data for all respondents
conjoint_data <- do.call(rbind, lapply(1:n_peeps, sim_one))

# remove values unobservable to the researcher
conjoint_data <- conjoint_data[ , c("resp", "task", "brand", "ad", "price", "choice")]

# clean up
rm(list=setdiff(ls(), "conjoint_data"))
\end{verbatim}

\end{tcolorbox}

\subsection{3. Preparing the Data for
Estimation}\label{preparing-the-data-for-estimation}

The ``hard part'' of the MNL likelihood function is organizing the data,
as we need to keep track of 3 dimensions (consumer \(i\), covariate
\(k\), and product \(j\)) instead of the typical 2 dimensions for
cross-sectional regression models (consumer \(i\) and covariate \(k\)).
The fact that each task for each respondent has the same number of
alternatives (3) helps. In addition, we need to convert the categorical
variables for brand and ads into binary variables.

\emph{todo: reshape and prep the data}

\subsection{4. Estimation via Maximum
Likelihood}\label{estimation-via-maximum-likelihood}

\emph{todo: Code up the log-likelihood function.}

\emph{todo: Use \texttt{optim()} in R or \texttt{scipy.optimize()} in
Python to find the MLEs for the 4 parameters (\(\beta_\text{netflix}\),
\(\beta_\text{prime}\), \(\beta_\text{ads}\), \(\beta_\text{price}\)),
as well as their standard errors (from the Hessian). For each parameter
construct a 95\% confidence interval.}

\subsection{5. Estimation via Bayesian
Methods}\label{estimation-via-bayesian-methods}

\emph{todo: code up a metropolis-hasting MCMC sampler of the posterior
distribution. Take 11,000 steps and throw away the first 1,000,
retaining the subsequent 10,000.}

\emph{hint: Use N(0,5) priors for the betas on the binary variables, and
a N(0,1) prior for the price beta.}

\_hint: instead of calculating post=lik*prior, you can work in the
log-space and calculate log-post = log-lik + log-prior (this should
enable you to re-use your log-likelihood function from the MLE section
just above)\_

\emph{hint: King Markov (in the video) use a candidate distribution of a
coin flip to decide whether to move left or right among his islands.
Unlike King Markov, we have 4 dimensions (because we have 4 betas) and
our dimensions are continuous. So, use a multivariate normal
distribution to pospose the next location for the algorithm to move to.
I recommend a MNV(mu, Sigma) where mu=c(0,0,0,0) and sigma has diagonal
values c(0.05, 0.05, 0.05, 0.005) and zeros on the off-diagonal. Since
this MVN has no covariances, you can sample each dimension independently
(so 4 univariate normals instead of 1 multivariate normal), where the
first 3 univariate normals are N(0,0.05) and the last one if
N(0,0.005).}

\emph{todo: for at least one of the 4 parameters, show the trace plot of
the algorithm, as well as the histogram of the posterior distribution.}

\emph{todo: report the 4 posterior means, standard deviations, and 95\%
credible intervals and compare them to your results from the Maximum
Likelihood approach.}

\subsection{6. Discussion}\label{discussion}

\emph{todo: Suppose you did not simulate the data. What do you observe
about the parameter estimates? What does
\(\beta_\text{Netflix} > \beta_\text{Prime}\) mean? Does it make sense
that \(\beta_\text{price}\) is negative?}

\emph{todo: At a high level, discuss what change you would need to make
in order to simulate data from --- and estimate the parameters of --- a
multi-level (aka random-parameter or hierarchical) model. This is the
model we use to analyze ``real world'' conjoint data.}




\end{document}
